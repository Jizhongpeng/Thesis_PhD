\chapter{Fresnel Simulation Toolbox: Theoretical Prerequisites}
\markboth{\MakeUppercase{Fresnel Simulation Toolbox: Theoretical Prerequisites }}{}
\label{chap:fresnel}
\section{Introduction}
\label{sec:intro}
This chapter describes a MATLAB simulation toolbox developed as a part of this doctorate project in order to investigate the potential of plenoptic imaging for biomedical and microscopy applications with particular interest in its behaviour at the diffraction limit. In literature are present many works on simulating plenoptic imaging devices. Plenoptic systems have been studied through simulation extensively. Until recently these studies have focussed on a geometrical optics approach, however in this project the focus is on investigating plenoptic systems biomedical and microscopy applications for which the performance at high resolution needs to be studied; hence diffraction cannot be ignored. Over the duration of this project there have been other research groups that have studied plenoptic systems under wave optics, including a wave optics analysis of plenoptic 1.0 systems, such as Schroff \textit{et al.} \cite{shroff2012high,shroff2013image} and Trujillo-Servilla \textit{et al.} \cite{birch2012depth} and a Fourier optics approach of the diffraction limit of a digital camera, Farrell \textit{et al.} \cite{farrell2012digital}. The growing interest in this aspect of plenoptic systems over the past few years is a product of the increasing interest in plenoptic systems overall and its new potential applications. In this work a Fresnel optics approach has been applied to a plenoptic camera with the purpose of understanding how the light field is recorded and to define some guidelines to design a working setup. Particular attention has been given to the behaviour of the system at its diffraction limit, a subject that has never been explicitly treated in literature. In this chapter four different methods to simulate light propagation in an optical system will be discussed and their performance compared in term of accuracy of the results, noise and computational effort.
A Fresnel optics approach has been chosen because the simulation tool to describe a plenoptic imaging system needs to have the following requirements:
\begin{itemize}
	\item it has to preserve the phase of the optical field propagating in order to preserve directional information of the rays of light;
	\item it has to take into account the effects of diffraction;
	\item it has to be adaptable in order to easily change the characteristic of the optical system, trying various configurations, keeping the operator based approach of ray tracing.
\end{itemize}
To address these features, a wave optics simulation toolbox has been designed and developed. Most of the existing literature on simulating plenoptic system are based on ray tracing techniques \cite{thurow2013recent,lynch2011development,lynch2012three,ng2006digital,levoy1996light}. The advantages of using ray tracing is that the transformations that rays undergo during the propagation can be described by two basic linear operators and compositions of them. These are the free space propagation and the lens operator. In developing the wave optics simulation this principle has been kept, and two operators have been defined: propagation and lens. Four different types of propagation operators have been described and compared. In analogy with ray tracing an optical system has been simulated using compositions of the two simple operators. In developing this platform all the media composing the system have been considered as linear, isotropic, homogeneous and non dispersive. 
\section{Scalar Theory of Diffraction}
\label{sec:scalar}
The term diffraction has been defined by Sommerfeld as any deviation of light rays from rectilinear paths which cannot be interpreted as reflection or refraction. \cite{sommerfeld1954optics}.The complex vectorial equations describing wave propagation in three dimensional space can be simplified into a set of scalar equations using the Scalar Theory of Diffraction as explained by Goodman \cite{goodman2005introduction}.
The starting point is given by Maxwell's Equations in absence of sources of electrical field of magnetic dipoles:\\
\begin{equation}
\label{eq:maxwell}
\begin{matrix}
	\nabla\times\overrightarrow{E}=-\mu\dfrac{\partial\overrightarrow{H}}{\partial t}\\
	\\
	\nabla\times\overrightarrow{H}=\epsilon\dfrac{\partial\overrightarrow{E}}{\partial t}\\
	\\
	\nabla\cdot\epsilon\overrightarrow{E}=0\\
	\\
	\nabla\cdot\mu\overrightarrow{H}=0\\
	\end{matrix}
\end{equation}
where $\overrightarrow{E}$ is the electric field, $\overrightarrow{H}$ is the magnetic field, $\mu$ and $\epsilon$ are respectively the magnetic permeability and electrical permittivity of the medium in which the optical wave is propagating. Both $\overrightarrow{E}$ and $\overrightarrow{H}$ are a functions of the position $x,y$ and $z$, as well as the time $t$. \\The operator $\nabla$ is defined as:\\
\begin{equation}
\label{eq:nabla}
	\nabla=\dfrac{\partial}{\partial x}\widehat{i}+\dfrac{\partial}{\partial y}\widehat{j}+\dfrac{\partial}{\partial z}\widehat{k}
\end{equation}
where $\widehat{i},\widehat{j}$ and $\widehat{k}$ are unit vectors along directions $x,y$ and $z$.\\
Propagation is assumed to happen in a dielectric medium that is linear, isotropic and homogeneous. A medium is linear if its response to a several disturbances acting simultaneously can be decomposed into the sum of the responses to the single disturbances taken individually. It is isotropic if its properties do not depends on the directions of polarization of the wave and is homogeneous if its permittivity is constant along all direction of propagation. The medium is considered also to be non dispersive, that is the permittivity $\epsilon$ is not dependent on the wavelength. \\
Applying the operator $\nabla\times$ to the left and to the right side of the first equation of \ref{eq:maxwell} and using the vector identity\\
\begin{equation}
\label{eq:vectidentity}
\nabla\times(\nabla\times\overrightarrow{E})=\nabla (\nabla\cdot\overrightarrow{E})-\nabla^2\overrightarrow{E}
\end{equation}
\begin{equation}
\label{eq:wave1}
	\nabla (\nabla\cdot\overrightarrow{E})-\nabla^2\overrightarrow{E}=\nabla\times(-\mu\dfrac{\partial\overrightarrow{H}}{\partial t})\\
\end{equation}
From the third equation in \ref{eq:maxwell}:\\
\begin{equation}
\label{eq:divergenza}
\nabla\cdot\epsilon\overrightarrow{E}=0\\
\end{equation}
hence equation \ref{eq:wave1} becomes:\\
\begin{equation}
\label{eq:wave2}
-\nabla^2\overrightarrow{E}=\nabla\times(-\mu\dfrac{\partial\overrightarrow{H}}{\partial t})\\
\end{equation}
since both the operators $\nabla\times$ and its derivative are linear it is possible to swap them on the right hand side of equation \ref{eq:wave1}.
Then substituting the second Maxwell equation \ref{eq:maxwell} into the equation \ref{eq:wave2}:
\begin{equation}
\label{eq:wave3}
-\nabla^2\overrightarrow{E}=-\mu\epsilon\dfrac{\partial^2}{\partial t^2}\overrightarrow{E}	
\end{equation}
Where $\nabla^2$ is the Laplacian operator defined as:
\begin{equation}
	\label{eq:laplacian}
	\nabla^2=\dfrac{\partial^2}{\partial x^2}\widehat{i}+\dfrac{\partial^2}{\partial y^2}\widehat{j}+\dfrac{\partial^2}{\partial z^2}\widehat{k}	
\end{equation}
The refractive index of the medium in which the wave is propagating is:\\
\begin{equation}
\label{eq:n}
n=\sqrt{\dfrac{\epsilon\mu}{\epsilon_0\mu_0}}
\end{equation} 
where $\epsilon_0$ is the permittivity of the vacuum and $\mu_0$ the magnetic permeability in vacuum. Therefore the speed of light in the vacuum is:
\begin{equation}
\label{eq:speedoflight}
c=\sqrt{\dfrac{1}{\epsilon_0\mu_0}}
\end{equation} 
Then the wave equation for the electric field becomes:
\begin{equation}
\label{eq:wave_elec}
\nabla^2\overrightarrow{E}-\dfrac{n^2}{c^2}\dfrac{\partial^2\overrightarrow{E}}{\partial t^2}=0
\end{equation}
Similar considerations can be done for the magnetic field, leading to an identical equation:
\begin{equation}
\label{eq:wave_magn}
\nabla^2\overrightarrow{H}-\dfrac{n^2}{c^2}\dfrac{\partial^2\overrightarrow{H}}{\partial t^2}=0
\end{equation}
Since the wave equation is obeyed by both the electric field and magnetic fields, it is possible to define a scalar wave equation, obeyed by the single components of those vectors. The scalar field components are represented as a function $u(x,y,z,t)$ called field disturbance.
The scalar wave equation is then:
\begin{equation}
\label{eq:scalar_wave}
\nabla^2u-\dfrac{n^2}{c^2}\dfrac{\partial^2u}{\partial t^2}=0
\end{equation}
With this scalar approximation it is possible to treat the propagation of an optical field as a scalar. This is only valid under the assumption of a linear, isotropic, homogeneous and non dispersive medium, since all the component in all the directions of the electric and magnetic fields must behave identically.
\subsection{Helmholtz Equation}
In the case of monochromatic waves the scalar field $u$ is a function of time $t$ and position $\overrightarrow{X}$ defined as:
\begin{equation}
\label{eq:scalarfield}
u(\overrightarrow{X},t)=A(\overrightarrow{X})cos[2\pi\nu t-\phi(\overrightarrow{X})]
\end{equation}
Where $A(\overrightarrow{X})$ is the amplitude of the disturbance and $\phi(\overrightarrow{X})$ is its phase at a point in the space with coordinates $\overrightarrow{X}=(x,y,z)$.
Separating space and time dependence:
\begin{equation}
\label{eq:scalarfield2}
u(\overrightarrow{X},t)=Re\{{U(\overrightarrow{X})e^{-j2\pi\nu t}}\}
\end{equation}
where $U$ is a complex function of position and includes the phase term $e^{j\phi(\overrightarrow{X})}$.
\begin{equation}
\label{eq:scalarfield3}
U(\overrightarrow{X})=A(\overrightarrow{X})e^{j\phi(\overrightarrow{X})}
\end{equation}
This field should satisfy the scalar wave equation \ref{eq:scalar_wave}. Substituting equation \ref{eq:scalarfield3} into \ref{eq:scalar_wave}:
\begin{equation}
\label{eq:scalar1}
	\nabla^2[U(\overrightarrow{X})e^{-j2\pi\nu t}]-\dfrac{n^2}{c^2}\dfrac{\partial^2}{\partial t^2}[U(\overrightarrow{X})e^{-j2\pi\nu t}]=0
\end{equation}
Expressing the derivatives and simplifying the exponential terms:
\begin{equation}
\label{eq:scalar2}
\nabla^2[U(\overrightarrow{X})]+ \left(\dfrac{2\pi\nu n}{c}\right)^2[U(\overrightarrow{X})]=0
\end{equation}
The wave number is defined as:
\begin{equation}
\label{eq:wavenumber}
	k=\dfrac{2\pi\nu n}{c}
\end{equation}
and expression \ref{eq:scalar2} becomes:
\begin{equation}
\label{eq:helmotz}
(\nabla^2+k^2)U=0	
\end{equation}
Equation \ref{eq:helmotz} is the Helmholtz equation and it describes the behaviour of a complex disturbance propagating in a homogeneous medium.
\subsection{Solutions of Helmholtz Equations}
 An analytical expression of the complex disturbance $U$ that satisfies the Helmholtz equation can be found using the Green's theorem under particular boundary conditions as explained by Goodman \cite{goodman2005introduction}. There are two possible solutions: Fresnel-Kirchhoff and Rayleigh-Sommerfeld.
Considering a wave $U_0$ propagating through a diffracting screen at a point in space with coordinates $z=0$ with an aperture $D$ the boundaries conditions:
\begin{itemize}
	\item Fresnel-Kirchhoff (FK) conditions \\
	\begin{math}
		\label{eq:FKbound}
		U(x,y;0)=U_0(x,y;0) \quad for \quad (x,y)\in D\\
		U(x,y;0)=0 \quad for \quad (x,y)\notin D\\
		\\
		\dfrac{\partial U}{\partial z}=\dfrac{\partial U_0}{\partial z} \quad for \quad (x,y)\in D\\
		\\
		\dfrac{\partial U}{\partial z}=0 \quad for \quad (x,y)\notin D\\	
		\end{math}
	\item Rayleigh-Sommerfeld (RS) conditions:\\
	\begin{math}
	U(x,y;0)=U_0(x,y;0) \quad for \quad (x,y)\in D\\
	U(x,y;0)=0 \quad for \quad (x,y)\notin D\\
	\end{math}
\end{itemize}
	The FK conditions lead to a simple result but they are not completely physically correct since they imply the field after the screen to be zero outside of the aperture in the immediate proximity of the screen as well as its normal derivative. The continuity of the normal derivative can lead to an unrealistic physical condition might lead to a situation where light is present even where it is not supposed to be in order to keep the optical field function smooth enough. Results given by the FK condition are accurate only for a distance from the aperture much larger than the wavelength.
	The RS condition on the other hand is less strict since it does not requires the derivatives of the disturbance. The only request is that the field is continuous along the aperture and this is more similar to the physical reality of the problem. It leads to a solution of the Helmholtz equation that is:
	\begin{equation}
	\label{eq:RSintegral}
	U(x,y;z)=\dfrac{1}{j\lambda}\iint_{\sigma}^{}U(\xi,\eta;0)\dfrac{e^{jkr}}{r} \cos(\theta) d\xi d\eta
		\end{equation}
		where, with reference to figure \ref{fig:RS}, $\theta$ is the angle between the z axis and the direction of propagation, $r=\sqrt{z^2+(x-\xi)^2+(y-\eta)^2}$ is the distance between the point $P_1=(x,y,z)$ and $P_0=(\xi,\eta;0)$ and $\sigma$ is the area of aperture. Since $\cos(\theta)=\dfrac{z}{r}$:
		\begin{equation}
		\label{eq:RSintegral1}
		U(x,y;z)=\dfrac{z}{j\lambda}\int\int_{\sigma}^{}U(\xi,\eta;0)\dfrac{e^{jkr}}{r^2} d\xi d\eta
		\end{equation}
		Equation \ref{eq:RSintegral1} is the Rayleigh-Sommerfeld diffraction formula \cite{goodman2005introduction}, and can be simplified under the Fresnel approximation, as will be explained in section \ref{sec:fresnelapprox}.
		\begin{figure}[H]
			\begin{center}
				\begin{tabular}{c}
					\includegraphics[height=5cm]{RS.eps}
				\end{tabular}
			\end{center}
			\caption{ \label{fig:RS} 
				Geometry of the aperture. }
		\end{figure} 
\subsection{The Fresnel Approximation}
\label{sec:fresnelapprox}
It is possible to approximate the distance of propagation $r$ between $P_0$ and $P_1$ with its Taylor expansion up to the second order when the distance of propagation is much larger than the aperture of the pupil (paraxial approximation):
\begin{equation}
\label{eq:taylor}
r=\sqrt{1+\left(\dfrac{x-\xi}{z}\right)^2+\left(\dfrac{y-\eta}{z}\right)^2}\approx z\left[1+\frac{(1)}{2}\left(\dfrac{x-\xi}{z}\right)^2+\frac{1}{2}\left(\dfrac{y-\eta}{z}\right)^2\right]
\end{equation}
Therefore for large propagation distances, $z\gg x,y$, the diffraction integral becomes:
\begin{equation}
\label{eq:Fresnel}
	U(x,y)=\dfrac{e^{jkz}}{j\lambda z} \int\int_{-\infty}^{\infty}U(\xi,\eta)e^{\frac{jk}{2z}\left[(x-\xi)^2+(y-\eta)^2\right]} d\xi d\eta
\end{equation}
Factorizing the exponential term the disturbance becomes:
\begin{equation}
\label{eq:Fresnel1}
U(x,y)=\dfrac{e^{jkz}}{j\lambda z} e^{j\frac{k}{2z}(x^2+y^2)} \int\int_{-\infty}^{\infty}U(\xi,\eta)e^{\frac{jk}{2z}(\xi^2+\eta^2)}e^{\frac{-jk}{2z}(x\xi+y\eta)} d\xi d\eta
\end{equation}
This can be seen as the Fourier transform of the disturbance before the aperture $U(\xi,\eta)$ multiplied by a quadratic phase factor
 $e^{\frac{jk}{2z}(\xi^2+\eta^2)}$ \cite{goodman2005introduction}.
\section{Free Space Propagation Operator: Fresnel Approximation Approach}
\label{sec:Fresnel}
The first version of the free space propagation operator has been developed using the Fresnel Integral as written in equation \ref{eq:Fresnel1}.
The input disturbance $U(\xi,\eta)$ is considered to be illuminated by monochromatic light with wavelength $\lambda$. As stated in section \ref{sec:fresnelapprox} the Fresnel integral can be seen as the two dimensional Fourier transform of the input field $U(\xi,\eta)$ multiplied by a quadratic phase factor \cite{goodman2005introduction, sypek1995light}. This is going to be very useful from a computational point of view since it can be implemented with a fast The function $U'(\xi,\eta)$ is defined as:
\begin{equation}
	\label{eq:U1}
	U'(\xi,\eta)=U(\xi,\eta)e^{\frac{jk}{2z}(\xi^2+\eta^2)}
\end{equation}
The optical field at the plane $z$ is the product of the Fourier transform of $U'(\xi,\eta)$ with the phase term $e^{\frac{jk}{2z}(x^2+y^2)}$:
\begin{equation}
	\label{eq:FT1}
	U(x,y) = \dfrac{e^{ikz}}{i\lambda z}e^{\frac{ik}{2z}(x^2+y^2)}\mathcal{F}[U'(\xi,\eta)]
\end{equation}
where the spatial frequencies of the Fourier Transform can be correlated with the spatial coordinates $x$ and $y$ by the relation:
\begin{equation}
\label{eq:spacefreq}
\left\{
\begin{array}{l l}
 f_x=\frac{x}{\lambda z} \\
 f_y=\frac{y}{\lambda z}
\end{array} \right.\
\end{equation}
This method is computationally fast since it requires only one Fourier transform, and it is analytically correct. Particular attention should be given to the sampling of the optical fields U and U'. Because of the presence of the Fourier transform the coordinates of the input and output fields are not sampled in the same way, but are scaled by a factor that is proportional to the distance of propagation, as shown in equation \ref{eq:spacefreq} \cite{gonzalez2004digital}. Therefore the input and output planes have different sampling \cite{sypek1995light}.
In addition the multiplicative phase factor:
\begin{equation}
\label{eq:phase2}
e^{\frac{ik}{2z}(\xi^2+\eta^2)}
\end{equation}
presents rapid oscillation of the phase of the optical field for small variations of z, since z is at its denominator \cite{sypek1995light, matsushima2009band}. In order to avoid aliasing the input field requires a large sampling and this is achieved by zero padding the sampling window of the input field, with an increase of the digital resolution of the field. It is known that the computational effort of the FFT algorithm increases with the resolution as $O(n \log n)$, where $n$ is the number of samples of the input field and increasing the sampling resolution of the field \cite{bracewell1965fourier} leads to a long computational time. 
\subsection{Multi-Step Fresnel Propagation Operator}
\label{sec:fresnelmulti}
To overcome the scaling of the field and the large computational time required by the Fresnel propagation integral method a modified method has been developed. 
A multi step approach as the one explained by Sypek \cite{sypek1995light, sypek2009reply} and shown in figure \ref{fig:multistepfig} was used. 
\begin{figure}[h]
	\begin{center}
		\begin{tabular}{c}
				\includegraphics[height=6cm]{multistepfig.eps}
		\end{tabular}
	\end{center}
	\caption{\label{fig:multistepfig} To remove the scaling factor between the input and output fields, a multi step Fresnel approach has been developed. The field is propagated by unit of dz, the minimum distance to keep the sampling the same. } 
	\end{figure}
The reason to adopt the multi-step approach in this project is to remove the scaling factor between the input field and the output field that arise from the Fourier transform as shown in equation \ref{eq:spacefreq}, while Sypek developed a multi step propagation model to minimize the oscillations of the Fourier spectrum and to avoid large zero paddings.\\ In the Fourier domain, an optical field sampled by a $N \times N$ pixels window, and with squared pixels that are $dx$ wide the spatial frequency resolution is \cite{bracewell1965fourier,gonzalez2004digital}:
\begin{equation}
\label{eq:getz1}
df_x=\dfrac{1}{Ndx}
\end{equation}
 Therefore the pixel size in the image plane is according to equations \ref{eq:spacefreq}:
\begin{equation}
\label{eq:pixelsize}
d\xi=d\nu\lambda z
\end{equation} 
where $z$ is the propagation distance and $\lambda$ is the wavelength of the monochromatic wave. The condition to keep the same resolution both in the input field and the output field is:
\begin{equation}
\label{eq:getz}
d\xi=df_x\lambda z=dx
\end{equation}
Substituting equation \ref{eq:getz1} into equation \ref{eq:spacefreq}:
\begin{equation}
\label{eq:getz2}
\dfrac{1}{Ndx}=\frac{dx}{\lambda z}
\end{equation}
Then resolving for z the minimum propagation distance $dz$ to keep the same sampling both in the input and output fields is:
\begin{equation}
\label{eq:getz3}
dz=\frac{W^2}{N\lambda }
\end{equation}
and:
\begin{equation}
	\label{eq:getz4}
	z=\displaystyle\sum_{i=1}^{N} dz_i
\end{equation}
where $W=Ndx$ is the dimension in meters of the input field. Equation \ref{eq:getz3} gives the length of the single step in which the propagation distance $z$ should be divided in order to keep the same resolution.\\ Although the results obtained with this multi-step approach are correct there are some issues. The propagation distance should be a multiple of $dz $, and this is a very significant limitation, especially since in simulating plenoptic systems the distances need to be set precisely. Another issue regards the computational time. With the multi step approach the number of FFT performed increases with the steps, leading to a computational time \textit{N} times larger that the Fresnel Integral method. For these reasons the angular spectrum method as will be discussed in section \ref{sec:angular} has been adopted in all the simulations presented in this work. 
\section {Free Space Propagation Operator: Angular Spectrum of Plane Waves Approach}
\label{sec:angular}
 In the Fourier domain the input disturbance can be seen as formed by a set of plane waves travelling in different directions, the \textit{Angular Spectrum of Plane waves} representation of an optical field. In the next section the propagation operator as and its characteristic transfer function will be defined. Three versions of the angular spectrum operator will be presented, and performances of the three versions will be compared.
\subsection{Angular Spectrum of Plane Waves}
\label{sec:angular 2}
The disturbance $U(x,y;0)$ describing a monochromatic wave incident on a plane \textit{(x,y)} at the \textit{z=0} while travelling along the z direction has a Fourier transform given by:
\begin{equation}
\label{eq:AS1}
A(f_x,f_y;0)=\iint_{-\infty}^{\infty} U(x,y;0)e^{-j2\pi(f_x x+f_y y)}dx dy
\end{equation}
and $U(x,y;0)$ is equal to the inverse Fourier transform of its spectrum:
\begin{equation}
\label{eq:AS2}
U(x,y;0)=\iint_{-\infty}^{\infty} A(f_x,f_y;0)e^{j2\pi(f_x x+f_y y)}df_x df_y
\end{equation}
The physical meaning fo the equation \ref{eq:AS2} is that the disturbance $U(x,y;0)$ can be decomposed in the sum of elemental plane waves propagating in directions given by the wave vector $\overrightarrow{k}$ whose magnitude is $2\pi/\lambda$ and direction is given by its direction cosines $(\alpha,\beta,\gamma)$ \cite{goodman2005introduction,matsushima2009band}.
Dropping the temporal dependence the plane wave is then:
\begin{equation}
\label{eq:AS3}
p(x,y;z)=e^{-j\overrightarrow{k}\cdot\overrightarrow{r}}\\
\end{equation}
where
\begin{equation}
\label{eq:AS4}
 \overrightarrow{r}=x\widehat{i}+y\widehat{j}+z\widehat{k}\\
\end{equation}
and
\begin{equation}
\label{eq:AS5}
 \overrightarrow{k}=\dfrac{2\pi}{\lambda}(\alpha\widehat{i}+\beta\widehat{j}+\gamma\widehat{k})	\\
\end{equation}
The exponential becomes:
\begin{equation}
\label{eq:AS6}
p(x,y;z)=e^{-j\frac{2\pi}{\lambda}(\alpha x+\beta y)}e^{-j\frac{2\pi}{\lambda}(\gamma z)}
\end{equation}
The terms, $\alpha, \beta$ and $\gamma$ are the direction cosines of the wave vector $\overrightarrow{k}$ and they are related as:
\begin{equation}
\label{eq:AS7}
\gamma = \sqrt{1-\alpha^2+\beta^2}
\end{equation}
Therefore the complex exponential function in equation \ref{eq:AS1} can be seen as a plane wave with direction cosines
\begin{equation}
\label{eq:AS8}
\alpha=\lambda f_x, \ \ \beta=\lambda f_y, \ \ \gamma = \sqrt{1-(\lambda f_x)^2+(\lambda f_y)^2}
\end{equation}
The angular spectrum of plane waves of the disturbance $U(x,y;0)$ is the function:
\begin{equation}
\label{eq:AS9}
A\left(\frac{\alpha}{\lambda},\frac{\beta}{\lambda};0\right)=\iint_{-\infty}^{\infty} U(x,y;0)e^{-j2\pi(\frac{\alpha}{\lambda} x+\frac{\beta}{\lambda} y)}dx dy
\end{equation}
After a prorogation of $z$ the disturbance $U(x,y;z)$ can be written in the form of the angular spectrum in analogy with equation \ref{eq:AS2}:
\begin{equation}
\label{eq:AS10}
U(x,y;z)=\iint_{-\infty}^{\infty} A(f_x,f_y;z)e^{j2\pi(f_x x+f_y y)}df_x df_y
\end{equation}
where $f_x=\alpha/\lambda$ and $f_y=\beta/\lambda$.
To be a propagative disturbance, equation \ref{eq:AS10} should satisfy the Helmholtz equation \ref{eq:helmotz}:
\begin{equation}
\label{eq:AS11}
(\nabla^2+k^2)U=0	
\end{equation}
Substituting equation \ref{eq:AS10} into \ref{eq:AS11}:
 \begin{equation}
 \label{eq:AS12}
\dfrac{d^2}{dz^2}A(f_x,f_y;z)+\left(\dfrac{2\pi}{\lambda}\right)[1-(\lambda f_x)^2 + (\lambda f_y)^2 ]A(f_x,f_y;z)=0
 \end{equation}
 A solution of the differential equation \ref{eq:AS12} is:
 \begin{equation}
 \label{eq:AS13}
 A(f_x,f_y;z)=A(f_x,f_y;0)e^{j \frac{2\pi}{\lambda}\sqrt{1-(\lambda f_x)^2+(\lambda f_y)^2}}
 \end{equation}
 The propagative solution is the one where the spatial frequencies satisfy the condition:
 \begin{equation}
 \label{eq:AS14}
 (\lambda f_x)^2+(\lambda f_y)^2<1
 \end{equation}
 in this case the exponential term in equation \ref{eq:AS13} remains complex and the wave can propagate since it is an oscillating term. For the values of spatial frequencies that satisfy the condition:
 \begin{equation}
 \label{eq:AS15}
 (\lambda f_x)^2+(\lambda f_y)^2>1
 \end{equation}
 the exponent in the equation \ref{eq:AS13} becomes real, and the exponential is a decay term. The solution is no longer propagative and waves are called evanescent waves. It is interesting to see how the angular spectrum theory is more complete than the Fresnel approximation since it includes evanescent components too.\\
 Finally, the disturbance after a propagation in z can be expressed as a function of the disturbance $U(x,y;0)$ at the plane z=0:
 \begin{equation}
 \label{eq:AS16}
 U(x,y;z)=\int\int_{-\infty}^{\infty} A(f_x,f_y;0)e^{j \frac{2\pi}{\lambda}\sqrt{1-(\lambda f_x)^2-(\lambda f_y)^2}}e^{j2\pi(f_x x+f_y y)}df_x df_y
 \end{equation}
 The last equation enables a calculation of the output field $U(x,y;z)$ in terms of the input field and the propagation distance, under the approximation of a linear, isotropic, homogeneous and non dispersive medium.\\
 Because of the linearity of the problem, the propagation is considered as a linear system that maps the input disturbance $U(x,y;0)$ into the a new field distribution $U(x,y;z)$\cite{goodman2005introduction}. This linear system is characterized by a transfer function whose bandwidth is limited to the case of the propagative solution of equation \ref{eq:AS12}, excluding the evanescent waves. The angular spectrum of the output field can be rewritten as the product of the angular spectrum of the input field multiplied by the transfer function $H(f_x,f_y)$:
 \begin{equation}
 \label{eq:AS17}
 A(f_x,f_y;z) = A(f_x,f_y;0)\cdot H(f_x,f_y;z)
 \end{equation}
 The propagation is fully described by the transfer function:
 \begin{equation}
 \label{eq:AS18}
H(f_x,f_y;z)=\begin{cases}e^{j \frac{2\pi z}{\lambda}\sqrt{1-(\lambda f_x)^2-(\lambda f_y)^2}} & \quad \text{if } \sqrt{f_x^2+f_y^2}<\frac{1}{\lambda}\\ 0 & \quad \text{otherwise }\\ \end{cases} 
 \end{equation}
 The bandwidth can be represented as a circle in Fourier space. For frequencies smaller than $1/\lambda$ the transfer function introduces a shift in the spatial domain that is responsible for diffraction \cite{goodman2005introduction}.
 Results obtained with the angular spectrum method are similar to the ones obtained with the Fresnel approximation, but no scaling factor between the input and output field is introduced. 
 From a computational point of view, the angular spectrum operator is composed of 3 steps: 
 \begin{enumerate}
 	\item Fourier transform of the input field
 	\item Multiplication of the Fourier transform of the input field with the propagation transfer function in equation \ref{eq:AS18} 
 	\item Inverse Fourier transform of the product at step 2. The resultant field is the output disturbance after the propagation in free space. 
 \end{enumerate}
 The process can be seen in figure \ref{fig:ASflux}
 \begin{figure}[H]
 	\begin{center}
 		\begin{tabular}{c}
 				\includegraphics[height=1.5cm]{asflux.eps}
 				\end{tabular}
 		\end{center}
 			\caption{ \label{fig:ASflux} 
 				Structure of the operator free space propagation with the angular spectrum of plane waves method. The initial disturbance $U(x,y;0)$ is transformed into the angular spectrum $A(f_x,f_y;0)$ with a Fourier transform implemented by a FFT algorithm. The angular spectrum is multiplied by the propagation transfer function $H(f_x,f_y)$ and the resultant angular spectrum is inverse transformed into the output disturbance $U(x,y;z)$ }
 \end{figure}
 \subsection{Band Limited Angular Spectrum}
 The transfer function of the propagation in equation \ref{eq:AS18} is a complex exponential oscillating with a frequency depending by the propagation distance $z$. Figure \ref{fig:AStransfer} shows four different profiles of the transfer function for four propagation distances, \textit{1 mm, 2 mm, 10 mm and 20 mm}. Aliasing effects are evident since for a propagation distance of \textit{2 mm}. When the \textit{z} becomes large, the aliasing arises at low spatial frequencies, narrowing the useful bandwidth of the transfer function \cite{matsushima2009band}.
 The transfer function can be rewritten as:
 \begin{equation}
 	\label{eq:BL1}
 	H(f_x,f_y) = e^{j \phi(f_x,f_y)}
 \end{equation}
 where $\phi$ is the oscillating phase term as:
 \begin{equation}
 	\label{eq:BL2}
 	\phi(f_x,f_y)=\dfrac{2\pi}{\lambda}\sqrt{1-\lambda f_x^2-\lambda f_y^2}
 \end{equation}
 \newpage
 \begin{figure}[H]
 	\begin{center}
 		\begin{tabular}{c}
 			\includegraphics[height=16cm]{phaseH01new.eps}
 		\end{tabular}
 	\end{center}
 	\caption 	{ \label{fig:AStransfer} 
 		Cross section along of the phase of the transfer function of the angular spectrum. It is evident how increasing the propagation distance increases the oscillating frequency leading to aliasing. }
 \end{figure}
 \newpage
 Defining the local spatial frequencies of the transfer function \cite{goodman2005introduction,matsushima2009band} as the frequency of the phase oscillation $\nu_x$ and $\nu_y$ along $f_x$ and $f_y$:
 \begin{equation}
 	\label{eq:BL3}
 	\begin{cases} \nu_x=\dfrac{1}{2\pi}\dfrac{\partial}{\partial f_x}\phi(f_x,f_y)\\
 	\\
 	 \nu_y=\dfrac{1}{2\pi}\dfrac{\partial}{\partial f_y}\phi(f_x,f_y)\\ \end{cases}
 \end{equation}
 the local spatial frequencies become:
 \begin{equation}
 \label{eq:BL16}
 \begin{cases} \nu_x=-\dfrac{z}{\lambda}\dfrac{\lambda^2 f_x}{\sqrt{1-(\lambda f_x)^2}}\\
 \\
 \nu_y=-\dfrac{z}{\lambda}\dfrac{\lambda^2 f_y}{\sqrt{1-(\lambda f_y)^2}}\\ \end{cases}
 \end{equation}
 As stated by Matsushima and Shimobaba \cite{matsushima2009band}, if the input optical disturbance is sampled by an $N \times N$ sampling window with pixel size $dx$, the transfer function is sampled by units of spatial frequency equal to $df=1/(N dx)$. It is not a differential but a range of sampling units. To satisfy the Nyquist condition the sampling frequency should be at least the double of the bandwidth of the transfer function. For one direction in the local frequency space:
 \begin{equation}
 \label{eq:BL4}
 \dfrac{1}{df}\geq 2|\nu_x|
 \end{equation}
 Modifying the sampling of the transfer function, and of the input field, can lead to huge sampling windows and long computational times. In addition to that in practical applications the sampling interval is usually fixed. Therefore the condition on the maximum frequency range for which the transfer function is not aliased:
 \begin{equation}
 \label{eq:BL5}
 \dfrac{1}{df_x}\geq 2 z\dfrac{|f_x|}{|\sqrt{(\frac{1}{\lambda^2}+f_x)^2}|}\\ 
 \end{equation}
 Resolving the equation for $|f_x|$ :
 \begin{equation}
 \label{eq:BL6}
 |f_x|\leq\dfrac{1}{\lambda\sqrt{(2df_x z)^2+1}} = f_{max}
 \end{equation}
 Where $f_{max}$ is the maximum frequency of the transfer function without generating errors due to aliasing. Assuming the sampling of the optical field to be the same in both x and y direction the maximum bandwidth for the sampling in $f_y$ is equal to the bandwidth in $f_x $. Therefore:
 \begin{equation}
 \label{eq:BL7}
 \dfrac{1}{df_y}\geq 2|\nu_y|
 \end{equation}
 \begin{equation}
 \label{eq:BL8}
 \dfrac{1}{df_y}\geq 2 z\dfrac{|f_y|}{|\sqrt{(\frac{1}{\lambda^2}+f_y)^2}|}\\ 
 \end{equation}
 \begin{equation}
 \label{eq:BL9}
 |f_y|\leq\dfrac{1}{\lambda\sqrt{(2df_y z)^2+1}} = f_{max}
 \end{equation}
 To avoid aliasing then, the two dimensional transfer function should be limited to a range of frequencies defined by equation \ref{eq:BL8}.
 The expression of the output field will then be:
 \begin{equation}
 \label{eq:BL10}
 U(x,y;z)= \mathcal{F}^{-1}\left[A(f_x,f_y;0)H'(f_x,f_y;z)\right]
 \end{equation}
 where
 \begin{equation}
 \label{eq:BL11}
 H'(f_x,f_y;z)=H(f_x,f_y;z)rect\left(\dfrac{f_x}{f_{max}}\right)rect\left(\dfrac{f_y}{f_{max}}\right)
 \end{equation}
 where $rect$ is the rectangular function or normalized boxcar function.
 The MATLAB algorithm to implement the band limitation consists in multiplying the phase term of the propagation transfer function by a phase mask with the shape of a circle function of radius $f_{max}$ in the plane of the spatial frequencies. A circular phase mask has been chosen instead of the rectangular phase mask of equation \ref{eq:BL11} because it follows the geometry of the lens aperture, without introducing new spatial features that could effect the diffraction.\\ The resultant phase of the transfer function is shown in figure \ref{fig:bandlimitedH}, with the bandwidth for three different propagation distances.
 \begin{figure}[H]
 	\begin{center}
 		\begin{tabular}{c}
 				\includegraphics[height=15cm]{phaseHBLnew.eps}
 		\end{tabular}
 	\end{center}
 	\caption { \label{fig:bandlimitedH} 
 		Bandwidth of the transfer function for three different propagation distances. }
 \end{figure} 
 From a computational point of view the algorithm to implement the band limited angular spectrum method can be summarized in the following steps:
 \begin{enumerate}
		\item Computation of the angular spectrum of the input disturbance via a Fourier transform as shown in equation \ref{eq:AS2}.
		\item Estimation of the maximum bandwidth of the transfer function in order to avoid aliasing error using equation \ref{eq:BL6}.
		\item Multiplication of the phase of the transfer function with a circular phase mask with radius equal to the frequency obtained in the previous step.
		\item Multiplication of the angular spectrum with the band limited transfer function.
		\item Inverse Fourier transform of the product at step 4 
 \end{enumerate}
 The structure of the free space propagation operator implemented by the band limited angular spectrum can be seen in figure \ref{fig:BLAS} 
 \newpage
 \begin{figure}[H]
 	\begin{center}
 		\begin{tabular}{c}
 			\includegraphics[height=5cm]{asfluxBL.eps}
 		\end{tabular}
 	\end{center}
 	\caption	{ \label{fig:BLAS} 
 		Structure of the operator free space propagation with the angular spectrum of plane waves method in its band limited version. The initial disturbance $U(x,y;0)$ is transformed into the angular spectrum $A(f_x,f_y;0)$ with a Fourier transform implemented by a FFT algorithm. The angular spectrum is multiplied by the propagation transfer function $H(f_x,f_y)$ whose bandwidth has been limited according to equation \ref{eq:BL9}. The bandwidth of the transfer function depends on the sampling of the input field, the wavelength of the light $\lambda$ and the propagation distance.The resultant angular spectrum is inverse transformed into the output disturbance $U(x,y;z)$. \textit{N} is the sampling of thr input field, \textit{z} the propagation distance and $\lambda$ the wavelength. }
 \end{figure} 
 \subsection{Corrected Band Limited Angular Spectrum Method}
 \label{sec:angular3}
 The third method to implement the propagation operator with the angular spectrum is the corrected band limited angular spectrum method. The difference between this method and the band limited angular spectrum method presented in section \ref{sec:angular 2} is that the bandwidth of the transfer function is truncated to the value of the cut-off frequency of the free space propagation in case the Nyquist criterion requires a bandwidth is too narrow.\\
 For long propagation distances the propagation transfer function acts as a low pass filter on the spatial frequency components of the input signal. When its bandwidth is bigger than the cutoff frequency calculated with the band limited method to avoid digital aliasing there is a loss of resolution in the final image. A trade-off should therefore be found between the error due to the aliasing and the error due to the excessive bandwidth limitation. For imaging applications this is not an issue since usually the numerical apertures of the optical elements in the imaging system already limit the band passing from the input field to the output field. However it is useful to estimate the bandwidth of the free space propagation defining the maximum spatial frequency that is transferred from the object plane to the image plane. This value defines the amount of information transferred from the input field to the image plane.\\
 With reference to figure \ref{fig:maxfrequency} for a point in the input field, the range of spatial frequencies that are transferred to the output field are the ones whose direction cosines are contained into the angle that includes the sampling window of the output field. In the cases of the propagation distance much bigger than the sampling window, $z>>W $ like in most of the applications of this simulation toolbox the angle $\theta$ is equal to:
 \begin{equation}
 \label{eq:BL12}
 \theta \simeq\dfrac{W}{z}
 \end{equation}
 \begin{figure}[h]
 	\begin{center}
 		\begin{tabular}{c}
 				\includegraphics[height=7cm]{cutofffrequency.png}
 		\end{tabular}
 	\end{center}
 	\caption	{ \label{fig:maxfrequency} 
 		The maximum spatial frequency is linked to the dimension of the sampling window of the output field $w$ and the propagation distance $z$ }
 \end{figure} 
 The link between the direction cosine and the spatial frequency is, according to equation \ref{eq:AS8}
 \begin{equation}
 \label{eq:BL13}
 \theta=\lambda f_{cutoff},
 \end{equation}
 The cutoff frequency of a field sampled by a sampling window $W$ and after a propagation distance of $z$ is defined as:
 \begin{equation}
 	\label{eq:BL14}
 	\nu_{cutoff}=\dfrac{W}{\lambda z}
 \end{equation}
 Therefore the bandwidth of the transfer function should be bigger than $\nu_{cutoff}$ in order not to introduce error in the reconstruction of the diffraction pattern due to loss in resolution.
 \begin{equation}
 \label{eq:BL15}
 f_{max}\geq\nu_{cutoff}
 \end{equation}
 If the propagation distance limits the bandwidth of the output field too much, the transfer function can be improved by increasing the sampling of the input field. This condition is equal to having a smaller pixel size. Dealing with bigger sampling windows however increases the computational effort. 

\section{Thin Lens Operator}
\label{sec:lens}
This section will describe how the thin lens operator has been designed and implemented in the simulation toolbox, following the method described in Goodman \cite{goodman2005introduction}.
A lens is composed of a material with a refractive index different form the one of air, that causes the propagation velocity of the optical field to drop. In the approximation of a thin lens, the translation of the ray of light inside the lens is negligible and if a ray enters the lens at the coordinate $(x,y)$ on one face, it exits at the same coordinates at the other side. A thin lens delays and incident wave-front by an amount proportional to its thickness. These delay can be modelled introducing a phase factor to the incident field. 
The thickness of the lens, $\Delta(x,y)$, is a function of the coordinates \textit{(x,y)} as shown in figure \ref{fig:thick}, where $\Delta_0$ is the lens maximum thickness at the coordinates $(0,0)$.
\begin{figure}[h]
	\begin{center}
		\begin{tabular}{c}
			\includegraphics[height=5cm]{lens1.eps}
		\end{tabular}
	\end{center}
	\caption{ \label{fig:thick} 
	Thickness of a lens as a function of position along x coordinate. It can be defined in the same way along the y coordinate}
\end{figure} 
The phase delay introduced by the lens is proportional to its thickness:
\begin{equation}
\label{eq:phase}
	\phi(x,y)=kn\Delta(x,y)+k[\Delta_0-\Delta(x,y)]
\end{equation} 	
where $n$ is the refractive index of the lens material and $k=2\pi/\lambda$ is the wave number. With reference to figure \ref{fig:thick} the quantity
\begin{math}
	kn\Delta(x,y)
\end{math}
is the phase delay introduced by the different material of the lens, and 
\begin{math}
	k[\Delta_0-\Delta(x,y)]
\end{math}
is the phase delay introduced by the free space propagation between the two planes represented with dashed line.
This phase delay can be seen as a multiplicative phase term defined as:
\begin{equation}
\label{eq:delay}
	t_{lens}=\exp[jk\Delta_0]exp[jk(n-1)\Delta(x,y)]
\end{equation}
The complex field 
\begin{math}
	U'(x,y)
\end{math}
immediately after the lens is therefore given by the product of the field entering the lens 
\begin{math}
	U_0(x,y)
\end{math} 
with the phase delay in equation \ref{eq:delay}.
\begin{equation}
\label{eq:disturbance}
	U'(x,y)=t_{lens}(x,y)U_0(x,y)
\end{equation}
The sign convention used in the following derivation for a ray of light travelling form left to right is: 
\begin{itemize}
	\item any convex surface encountered has positive radius of curvature
	\item any concave surface encountered had negative radius of curvature
\end{itemize}
The lens can be split into three elements such that the total thickness function is the sum of three functions:
\begin{figure}[h]
	\begin{center}
		\begin{tabular}{c}
				\includegraphics[height=5cm]{lens2.eps}
		\end{tabular}
	\end{center}
	\caption{ \label{fig:split} 
		The thickness function can be decomposed into the sum of three contributions. }
\end{figure} 
From figure \ref{fig:split}:
\begin{equation}
\label{eq:split}
	\Delta(x,y)=\Delta_1(x,y)+\Delta_2(x,y)+\Delta_3(x,y)
\end{equation}
Where the three components of the thickness are:
\begin{equation}
\label{eq:delta1}
\Delta_1(x,y)=\Delta_{01}(x,y)-R_1\bigg(1-\sqrt{1-\frac{x^2+y^2}{R_1^2}}\bigg)
\end{equation}
\\
\begin{equation}
\label{eq:delta2}
\Delta_2(x,y)=\Delta_{02}
\end{equation}
\\
\begin{equation}
\label{eq:delta3}
\Delta_3(x,y)=\Delta_{03}(x,y)+R_2\bigg(1-\sqrt{1-\frac{x^2+y^2}{R_2^2}}\bigg)
\end{equation}
\\

\begin{math}
	R_1>0
\end{math}
and
\begin{math}
	R_2<0
\end{math}
are respectively the radius of curvature of the right-hand surface and the radius of curvature of the left-hand side of the lens surface.
Then total thickness function is given by:
\begin{equation}
\label{eq:delta_sum}
	\Delta(x,y)=\Delta_0(x,y)-R_1\bigg(1-\sqrt{1-\frac{x^2+y^2}{R_1^2}})+R_2(1-\sqrt{1-\frac{x^2+y^2}{R_2^2}}\bigg)
\end{equation}
where $\Delta_0$ is:
\begin{math}
	\Delta_0=\Delta_{01}+\Delta_{02}+\Delta_{03}
\end{math}.
In the paraxial approximation it is possible to approximate $\Delta(x,y)$ with its first element of the Taylor series:
\begin{equation}
\label{eq:approx1}
\sqrt{1-\frac{x^2+y^2}{R_1^2}}\approx 1-\frac{x^2+y^2}{R_1^2}
\end{equation}
\begin{equation}
\label{eq:approx2}
\sqrt{1-\frac{x^2+y^2}{R_2^2}}\approx 1-\frac{x^2+y^2}{R_2^2}
\end{equation}
 Substituting \ref{eq:approx1} and \ref{eq:approx2} into equation \ref{eq:delta_sum}:
 \begin{equation}
 \label{eq:approx3}
 \Delta(x,y)=\Delta_0(x,y)-\frac{x^2+y^2}{2}\bigg(\frac{1}{R_1}+\frac{1}{R_2}\bigg)
 \end{equation}
and substituting \ref{eq:approx3} into equation \ref{eq:phase} the result is the expression of the phase shift in the exponential form:
 \begin{equation}
 \label{eq:expon}
 t_{lens}(x,y)=\exp[jkn\Delta_0]exp[-jk(n-1)\frac{x^2+y^2}{2} \bigg(\frac{1}{R_1}+\frac{1}{R_2}\bigg)]
 \end{equation}
 The focal length of the lens $f$ is the quantity containing the information on the physical properties of the lens and it is defined as:
	\begin{equation}
 \label{eq:focal}
 \frac{1}{f} \equiv(n-1) \bigg(\frac{1}{R_1}+\frac{1}{R_2}\bigg)
	\end{equation}
 Finally after dropping the constant phase factor the effects of a thin lens under paraxial approximation as a quadratic phase transformation are described by equation \ref{eq:lens1}:
 \begin{equation}
 	\label{eq:lens1}
 	t_{lens}(x,y)=\exp[-j\frac{k}{2f}(x^2+y^2)]
 \end{equation}
 This equation can represent the effects of any lens, since the sign of $f$ will define whether the lens is positive or negative.
 The physical meaning of this expression can be understood through figure \ref{fig:lensmeaning}.
	\begin{figure}[h]
		\begin{center}
			\begin{tabular}{c}
					\includegraphics[height=7cm]{lens3.eps}
			\end{tabular}
		\end{center}
		\caption{ \label{fig:lensmeaning} 
			A thin lens converts a plane wave into a spherical wave. }
	\end{figure} 
	When a plane wave represented by the complex field $U(x,y)$ is normally incident on the lens, according to equation \ref{eq:disturbance}, the field leaving of the lens will be:
	\begin{equation}
	\label{eq:disturbance2}
		U'(x,y)=\exp[-j\frac{k}{2f}(x^2+y^2)]
	\end{equation}
	The thin lens is introducing a phase term on the incident field, and the resulting field is a quadratic approximation of a spherical wave. Figure \ref{fig:lensmeaning} shows that when $f$ is positive, the wave is converging towards a point on the optical axis at distance $f$ from the lens. If $f$ is negative the wave is diverging from a point on the optical axis placed at a distance $f$ in front of the lens. The lens therefore transforms a plane wave into a spherical wave. 
	The operator thin lens is defined as the product of the quadratic transformation in equation \ref{eq:lens1} with a function describing the aperture of the lens, called pupil $P(x,y)$ :
	\begin{equation}
	\label{eq:lesop}
		L(x,y)=P(x,y)\exp[-j\frac{k}{2f}(x^2+y^2)]
	\end{equation}
	where the pupil function is defined as:
	\begin{equation}
	\label{eq:pupil}
	 P(x,y) = \left\{
		\begin{array}{l l}
		1 & \quad \sqrt{x^2+y^2}<r \\
		0 & \quad \text{elsewhere}
		\end{array} \right.\
	\end{equation}
	where $r$ is the radius of the lens aperture.
	\subsection{Aliasing in Phase Sampling}
	The quadratic phase factor of the thin lens operator is a complex function whose modulus is constant and equal to unity, and its argument is an oscillating function in the interval $(0,2\pi]$. The argument of the phase factor is shown in figure \ref{fig:argument}
	\begin{figure}[H]
		\begin{center}
			\begin{tabular}{c}
					\includegraphics[height=7cm]{C:/Users/Massimo/Documents/Thesis/Thesis_PhD/phaselens.eps}
			\end{tabular}
		\end{center}
		\caption{ \label{fig:argument} 
			Phase profile of a computer generated lens. The blue line shows the quadratic phase term and the red line shows the same phase term wrapped every $2\pi$. This wrapping is the source of the aliasing. }
	\end{figure} 
The parabolic phase profile gets wrapped since the argument of a complex function varies between $0$ and $2\pi$. This wrapping of the phase can cause aliasing as the distance from the centre of the lens increases. For high values of $x$ and $y$, the quadratic term becomes too steep and the frequency of oscillation between $0$ and $2\pi$ of the phase factor might become too high to be correctly sampled. With analogy of what has been said in section \ref{sec:angular}, for a parabolic phase profile:
	\begin{equation}
		\label{eq:parabolic}
		\phi(x,y)=\dfrac{k}{2f}(x^2+y^2)
	\end{equation} 
The instantaneous frequencies of the phase $\nu_x$, $\nu_y$ in $x$ and $y$ respectively are defined as:
	 
\begin{equation}
\label{eq:localfreq}
\left\{
\begin{array}{l l}
\nu_x=\dfrac{1}{2\pi}\dfrac{\partial\phi}{\partial x}=\dfrac{1}{2\pi}\dfrac{k}{ f}x\\
\\
\nu_y=\dfrac{1}{2\pi}\dfrac{\partial\phi}{\partial y}=\dfrac{1}{2\pi}\dfrac{k}{ f}y
\end{array} \right.\
\end{equation}
In order to recover all the components of a periodic signal, the sampling frequency should be at least twice the bandwidth of the signal. According to the Nyquist criterion:
\begin{equation}
\label{eq:nyquist}
	f_{Nyquist}=2\nu_{max}
\end{equation} 
The maximum instantaneous frequency $\nu_{max}$ of the phase factor is the one that correspond to the edge of the lens. For brevity the one dimensional case along the $x$ direction is considered, but the conclusions are valid for the $y$ direction because of the rotational symmetry of the pupil function in equation \ref{eq:pupil}. In this case $x_{max}=y_{max}=r$ where $r$ is the radius of the pupil function.
Defining the pixel size $\Delta x$ along $x $, and assuming the pixels as square, so that the sampling frequency $1/\Delta x$ is the same along both axis, the Nyquist condition to avoid aliasing is:
\begin{equation}
	\label{eq:Nyx}
	\dfrac{1}{\Delta x}\geq \dfrac{1}{2\pi}\dfrac{k}{f}x_{max}
\end{equation}
Since in general the sampling rate cannot be modified because of the fixed pixel size of the sensor this expression leads to a condition for the radius of the pupil function:
\begin{equation}
\label{eq:radius_cond}
	r=x_{max}\leq \frac{2\pi f}{k \Delta x}
\end{equation}
Substituting $k=\frac{2\pi}{\lambda}$ into \ref{eq:radius_cond}:
\begin{equation}
\label{eq:radius_cond1}
r\leq \frac{\lambda f}{\Delta x}
\end{equation}
The more powerful the lens, $f$ will have a smaller value and hence the aperture should be smaller.
Therefore, in order to achieve a desired aperture size in equation \ref{eq:radius_cond1}, the focal length must be such that:
\begin{equation}
	\label{eq:focalcond}
	f\geq\frac{r\Delta x }{\lambda}
\end{equation}
and given that the digital resolution of a lens with aperture equal to $2r$ is given by:
\begin{equation}
\label{eq:resolution}
N=\frac{2r}{\Delta x}
\end{equation}
then the minimum resolution of a lens of radius $r$ and focal length $f$ is:
\begin{equation}
	\label{eq:rescond}
	N\geq\frac{2r^2}{f \lambda} 
\end{equation}
Applying these conditions will assure that no aliasing will be introduced by the lens operator.
\section{Conclusions}
This chapter described the theoretical basis of the wave optics simulation toolbox developed during this doctoral work.
From the Maxwell's equation following the theory found in literature it was shown how an optical field can be described as a scalar field solution of the Fresnel integral. This can be used to link together the field at a certain point in the space and the same field after a propagation along a certain direction, since the output field can be expressed in terms of th input field. This solution presents aliasing and scaling problems. A second method, already described in literature, is the multi step Fresnel propagation operator. It has been proposed to address the scaling issue and consists in dividing the propagation distance in smaller intervals such that no scaling is introduced. The problem with this method is the noise produced and the long computational time required. 
The third method studied to implement the free space propagation was the angular spectrum of plane waves. It produces exact results, without any scaling and requires a short computational time. The propagation is seen as a linear filter with a characteristic transfer function whose bandwidth is proportional to the propagation distance. Aliasing arise if the resolution of the optical field is not enough to sample correctly the transfer function. To avoid this a fourth method was introduced: the band limited angular spectrum. It was implemented following results already presented in literature. An original contribution to the field was given by the correction imposed on the band limitation of the band limited angular spectrum method, leading to the definition of a fifth method that has been used in all the simulation results presented in the following chapters. A trade-off between the noise produced by aliasing and the loss of information caused by the bandwidth truncation was defined and is was possible to conserve most of the information carried by the optical filed reducing the aliasing error. As it will be shown in the next chapter this last method is able to give the lowest signal to noise ratio.
This five propagation methods characteristics are resumed in table:
\begin{table}
	\centering
	\label{tab:methods}
	\begin{tabular} { l | p{4cm} | p{4cm}}
		Method & Advantages & Drawbacks\\ \hline
		Fresnel integral & Theoretically accurate & Noisy, Scaling factor \\ \hline
		Multi-step & No scaling & Noisy, slow \\ \hline
		Angular Spectrum & No scaling, fast & Aliasing for long propagations \\ \hline
		Band Limited AS & No scaling, fast, no aliasing & Loss of information \\ \hline
		Corrected Band limited AS & No scaling, fast, no aliasing, keeps information & Trade off between  noise reduction and information transmitted\\ \hline
		
		
	\end{tabular}
	\caption{Free space propagation methods}
\end{table}
	
After the free space propagation operator was defined, the thin lens operator was analysed. A thin lens is described by a pupil function with a quadratic phase profile. In analogy with what seen for the angular spectrum transfer function, also the lens phase profile can be under sampled causing aliasing in the output optical field. Therefore a set of design equation have been derived applying the Nyquist criterion to the lens phase profile.
The next chapter will show how this operators are practically used in the simulation toolbox.
 
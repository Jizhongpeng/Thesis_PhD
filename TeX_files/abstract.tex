\thispagestyle{plain}
\begin{center}
	\Large
	\textbf{Investigation of Plenoptic Imaging Systems: a Wave Optics Approach}
	\vspace{0.4cm}
	\\
	\large
	\textbf{Massimo Turola}
	
	\vspace{0.9cm}
	\textbf{Abstract}
\end{center}

Conventional imaging devices only capture a part of the total information carried by the light. A new generation of imaging devices, plenoptic systems, use an array of micro lenses to codify the light coming from an object into a four dimensional function called light field. The final image is then obtained after post processing computations on the light field. \\
 In this work plenoptic imaging devices are analysed using a wave optics approach. A platform to simulate light propagating under the Fresnel approximation in a generic optical system was developed in MATLAB. An optical system can be modelled as the composition of two basic operators: the free space propagation and the lens. The first one was implemented developing an original method derived from the angular spectrum of plane waves theory of propagation. The second was implemented using a phase mask. The code was developed to optimize the signal to noise ratio and the computational time.
 \\
Two different configurations of plenoptic imaging systems were simulated. The first is the plenoptic 1.0 configuration. The general theory of plenoptic 1.0 and the post processing algorithms presented in literature were verified using the simulation platform. The effects of diffraction were also evaluated and an original refocusing method is presented. For the second configuration, plenoptic 2.0, a full study of the optical resolution has been made and a detailed analysis of the effects of diffraction is presented. The results achieved with the simulations have been used to design a working prototype of plenoptic microscope.
\\
 This novel wave optics approach enables to quantify for the first time in literature the effects of diffraction on this class of devices. In plenoptic 1.0 diffraction is a source of noise due to the crosstalk between neighbouring lenslets. In plenoptic 2.0 systems the optical resolution is directly proportional to the magnification of the lenslet array. A small magnification leads to a high directional sampling but at the same time to a loss of optical resolution. The finite dimensions of the lenslets together with the wave nature of light produce a physical limit to the amount of information that can be achieved sampling the optical fields with those kind of devices.